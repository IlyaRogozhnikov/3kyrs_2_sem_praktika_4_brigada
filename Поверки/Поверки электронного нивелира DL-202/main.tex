\documentclass[a4paper]{article}
\usepackage[utf8]{inputenc}
\usepackage[14pt]{extsizes}
\usepackage[russian]{babel}
\usepackage{setspace,amsmath}
\usepackage[left=20mm, top=20mm, right=20mm, bottom=20mm]{geometry}
\setlength{\parindent}{12,5mm}
\linespread{1.15}
\usepackage{fontspec} 
\defaultfontfeatures{Ligatures={TeX},Renderer=Basic} 
\setmainfont[Ligatures={TeX,Historic}]{Times New Roman}
\usepackage{multirow} %вертикальное обьединение ячеек


\begin{document}

\begin{titlepage}
\large{

\begin{center}
Белорусский Национальный Технический Университет\\
Факультет Транспортных Коммуникаций\\
Кафедра <<Геодезия и аэрокосмические геотехнологии>>\\
~\\
~\\
~\\
~\\
~\\
~\\
Отчет\\
о выполнении поверок электронного нивелира DL-202\\
~\\
~\\
~\\
~\\
~\\
\end{center}

\begin{flushright}
Выполнил: Бригада №4\\
Греченков Т.А.\\
Гречный К.М.\\
Лабудев Н..\\
Прудников М..\\
Рогожников И.А.\\
Проверил ст. преподаватель\\
Будо А.Ю.
~\\
~\\
~\\
~\\
\begin{center}
Минск,2021
\end{center}
\end{flushright}
}
\end{titlepage}

\begin{newpage}
\large{
    \par\textbf{1. Поверка цилиндрического уровня}
    \par\textbf{Выполнение:} Вращая подъемные винты подставки нивелира, приводят пузырек круглого уровня в нуль-пункт. Затем поворачивают уровень вместе со зрительной трубой нивелира по азимуту на 180$^\circ$. Если при этом пузырек останется в центре ампулы, то условие выполнено. В противном случае, действуя исправительными винтами уровня, перемещают пузырек на половину дуги отклонения его от нуль-пункта в направлении к центру ампулы, а на вторую половину отклонения — при помощи подъемных винтов. После этого вновь производят поверку. Так поступают до тех пор, пока условие не будет выполнено.
    \par\textbf{Результат:} Поверка выполняется.
}
\large{
    \par\textbf{2. Вертикальная нить сетки должна совпадать с отвесом (быть параллельна вертикальной оси вращения нивелира).}
    \par\textbf{Выполнение:} Для выполнения этой поверки в защищенном от ветра месте подвешивают на тонком шнуре тяжелый отвес. На расстоянии 20—25 м от отвеса устанавливают поверяемый нивелир и приводят его в рабочее положение (при помощи элевационного винта совмещают видимые в поле зрения трубы концы пузырька цилиндрического уровня). Затем один конец вертикальной нити сетки совмещают с отвесом. Если другой конец этой нити отойдет от отвеса более, чем на 0,5 мм, то исправляют установку сетки в зрительной трубе. Для этого снимают окулярную часть зрительной трубы и отпускают винты, крепящие оправу пластинки с сеткой нитей к корпусу трубы. Затем, перемещая пластинку, устанавливают сетку в соответствующее положение, закрепляют винты и присоединяют окуляр, и вновь повторяют эту поверку.
    \par\textbf{Результат:} Поверка выполняется.
}
\\
\large{
    \par\textbf{3. Ось цилиндрического уровня должна быть параллельна визирной оси зрительной трубы.}
    \par\textbf{Выполнение:} Проверка этих условий выполняется двойным нивелированием пары точек способом "из середины" и "вперед"(рис.33). Для этого закрепляют неподвижно две нивелирные рейки на расстоянии 60-90 м, а нивелир устанавливают между ними на середину с погрешностью 1 м. Расстояния до реек измеряют нитяным дальномером. Определяют превышение между рейками при двух горизонтах прибора, как разность отсчетов на заднюю и переднюю рейки. Превышение, полученное при одном горизонте прибора, не должно отличаться от превышения, полученного при втором горизонте прибора, не более 3 мм. Затем выбирают вторую станцию на расстоянии предела фокусирования (2...3 м) от одной из реек и берут по ней отсчет. Используя этот отсчет и превышение, полученное на первой станции вычисляют отсчет по дальней рейке.
    \par\textbf{Результат:} Поверка выполняется.
}

\end{newpage}

\begin{newpage}
\begin{center}
\normalsize{
Журнал нивелирования 3 класса
\\
\begin{tabular}{|c|c|c|c|c|}
\hline
\multicolumn{5}{|c|}{Прямой ход}\\
\hline
Рейка & Задняя & Передняя & Превышение & Ср. превышение\\
\hline
Ч & 1,06    & 1,8908  & -0,7708 & \multirow{4}{*}{-0,7712}\\
\cline{1-4}
К & 0,9467  & 1,7183  & -0,7716 & \\
\cline{1-4}
  &         &         & 0,0008  & \\
  & -0,1133 & -0,1125 & 0,0008  & \\\hline
Ч & 2,0746  & 0,9630  & 1,1116  & \multirow{4}{*}{1,1112}\\
\cline{1-4}
К & 1,9216  & 0,8109  & 1,1107  & \\
\cline{1-4}
  &         &         & 0,0009  & \\
  & -0,153  & -0,1521 & 0,0009  & \\
\hline
Ч & 1,3968 & 1,7411   & -03443  & \multirow{4}{*}{-0,3441}\\
\cline{1-4}
К & 1,2243 & 1,5682   & -0,3439 & \\
\cline{1-4}
  &         &         & 0,0004  & \\
  & -0,1725 & -0,1729 & 0,0004  & \\
\hline
  & 8,624   & 8,6323  & -0,0083 & \multirow{2}{*}{-0,0041}\\
  &         &         & 0,000415& \\
\hline
  & -0,0083 &         & -0,0042 & \\
  & -0,00415&         &         & \\
\hline

\multicolumn{5}{|c|}{Обратный ход}\\
\hline
Рейка & Задняя & Передняя & Превышение & Ср. превышение\\
\hline
Ч & 1,7146  & 1,9699  & 0,3447  & \multirow{4}{*}{0,3450}\\
\cline{1-4}
К & 1,6117  & 1,2665  & 0,3452 & \\
\cline{1-4}
  &         &         & -0,0005 & \\
  & -0,1028 & -0,1034 & -0,0005 & \\
\hline
Ч & 1,0541  & 2,1665  & -1,1124 & \multirow{4}{*}{-1,1129}\\
\cline{1-4}
К & 0,9522  & 2,0656  & -1,1134 & \\
\cline{1-4}
  &         &         & -0,001  & \\
  & -0,1019 & -0,1009 & -0,001  & \\
\hline
Ч & 1,7776 & 1,0105   & 0,7671  & \multirow{4}{*}{-0,7674}\\
\cline{1-4}
К & 1,6259 & 0,8581   & 0,7678  & \\
\cline{1-4}
  &         &         & -0,0007 & \\
  & -0,1725 & -0,1729 & -0,0007 & \\
\hline
  & 8,7361  & 8,7371  & -0,001  & \multirow{2}{*}{-0,0005}\\
  &         &         & 0,000415& \\
\hline
  & -0,001 &         & -0,0042  & \\
  & -0,0005&         &          & \\
\hline
\end{tabular}
}
\end{center}
\end{newpage}

\end{document}
